\documentclass[a4paper]{ctexart}
\pagestyle{plain}
\usepackage{url}
\usepackage{listings}
\usepackage{enumerate}
\usepackage{graphicx}
\usepackage{subfigure}
\usepackage{float}
\usepackage{geometry}
\geometry{left=1in, right=1in, top=1in, bottom=1in}
\CTEXsetup[format={\large\bfseries}]{section}
\CTEXsetup[format={\bfseries}]{subsection}

\begin{document}
\bibliographystyle{unsrt}
\title{数字图像处理\quad 综合作业4}
\author{罗云鹏\ 自64\ 2016011470}
\maketitle

\section{算法原理}
对每个 $8\times 8$图像块进行DCT后,再使用Normalization Matrix进行归一化。
由于其直流分量比较大,对直流分量使用LPC单独编码存储,有助于提高压缩比。

当quality数值较大时,比如取20,得到的交流分量的值正则化后为0,没有需要进行霍夫曼编码的数值,这是matlab会报错。
在程序中检测待编码数值的个数,如果为0时则不进行霍夫曼编码,直接存储。

\section{结果及分析}
实验结果如下图所示。其中左为RGB通道分别压缩结果,右为转为YCbCr空间后,Y 通道图像按原始分辨率压缩,CbCr 通道图像缩小一半,分别进行压缩。

不难看出RGB三通道压缩时,图像质量受损较不明显。
而转到YCbCr空间后压缩,图像的压缩比提高,但图像质量受损明显,且当quality较大时,颜色有失真。

\begin{figure}[!htp]
    \begin{center}
        \subfigure[quality=1 rmse:4.7941 ratio:0.29856]{\includegraphics[width=.33\textwidth]{../pic/1rgb.png}}
        \subfigure[quality=1 rmse:6.2381 ratio:0.12984]{\includegraphics[width=.33\textwidth]{../pic/1ycbcr.png}}
        \subfigure[quality=5 rmse:8.1274 ratio:0.071585]{\includegraphics[width=.33\textwidth]{../pic/5rgb.png}}
        \subfigure[quality=5 rmse:11.0815 ratio:0.033019 ]{\includegraphics[width=.33\textwidth]{../pic/5ycbcr.png}}
        \subfigure[quality=10 rmse:11.1536 ratio:0.038572 ]{\includegraphics[width=.33\textwidth]{../pic/10rgb.png}}
        \subfigure[quality=10 rmse:16.4155 ratio:0.018884 ]{\includegraphics[width=.33\textwidth]{../pic/10ycbcr.png}}
        \subfigure[quality=20 rmse:16.7867 ratio:0.021161 ]{\includegraphics[width=.33\textwidth]{../pic/20rgb.png}}
        \subfigure[quality=20 rmse:26.4023 ratio:0.020331 ]{\includegraphics[width=.33\textwidth]{../pic/20ycbcr.png}}
    \end{center}
\end{figure}

\end{document}
