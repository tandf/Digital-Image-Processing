\documentclass[a4paper]{article}
\usepackage{xeCJK}
\usepackage{url}
\usepackage{listings}
\usepackage{graphicx}
\usepackage{subfigure}
\setCJKmainfont{STXihei}

\begin{document}
\bibliographystyle{unsrt}
\title{数字图像处理\quad 大作业1}
\author{罗云鹏\ 自64\ 2016011470}
\maketitle

具体操作见代码及注释。

\section{分块dft}
对图像分成大小为 $32 \times 32$的像素块,并做离散傅里叶变换,提取结果中幅值最高的点。
注意提取幅值最高点时应排除直流和低频部分。

\section{计算频率图和幅值图}
由傅里叶变换特性,根据幅值最高点相对于直流点的距离,可估算出脊线频率。
计算傅里叶变换结果中,幅值最高点的幅值。

对频率图进行低通滤波。

\section{前背景分离}
根据幅值和脊线频率判断指纹所在位置,得到背景蒙版,将背景区分开。

\section{计算方向图}
根据幅值最高点相对于直流点的方向,可计算出脊线方向。
根据傅里叶变换特性,此方向与实际方向垂直。
将方向转化为复数,便于使用低通滤波器对方向图进行平滑。

对方向图进行低通滤波

\section{使用 Gabor 滤波器增强指纹}
根据计算得到的方向和频率,对指纹部分图像做滤波,背景部分取0。

\section{算法结果}

\begin{figure}[!htp]
    \begin{center}
        \subfigure[原图]{ \includegraphics[width=0.45\textwidth]{../code/img1.png} }
        \subfigure[频率图]{ \includegraphics[width=0.45\textwidth]{../code/lf1.jpg} }
        \subfigure[幅值]{ \includegraphics[width=0.45\textwidth]{../code/lm1.jpg} }
        \subfigure[前背景分离蒙版]{ \includegraphics[width=0.45\textwidth]{../code/mask1.jpg} }
        \subfigure[方向图]{ \includegraphics[width=0.45\textwidth]{../code/lo1.jpg} }
        \subfigure[脊线增强结果]{ \includegraphics[width=0.45\textwidth]{../code/filtered1.jpg} }
    \end{center}
    \caption{FTIR处理结果}
    \label{fig:img1}
\end{figure}

\begin{figure}[!htp]
    \begin{center}
        \subfigure[原图]{ \includegraphics[height=0.3\textheight]{../code/img2.png} }
        \subfigure[频率图]{ \includegraphics[height=0.3\textheight]{../code/lf2.jpg} }
        \subfigure[幅值]{ \includegraphics[height=0.3\textheight]{../code/lm2.jpg} }
        \subfigure[前背景分离蒙版]{ \includegraphics[height=0.3\textheight]{../code/mask2.jpg} }
        \subfigure[方向图]{ \includegraphics[height=0.3\textheight]{../code/lo2.jpg} }
        \subfigure[脊线增强结果]{ \includegraphics[height=0.3\textheight]{../code/filtered2.jpg} }
    \end{center}
    \caption{phone处理结果}
    \label{fig:img2}
\end{figure}

\begin{figure}[!htp]
    \begin{center}
        \subfigure[原图]{ \includegraphics[width=0.45\textwidth]{../code/img3.png} }
        \subfigure[频率图]{ \includegraphics[width=0.45\textwidth]{../code/lf3.jpg} }
        \subfigure[幅值]{ \includegraphics[width=0.45\textwidth]{../code/lm3.jpg} }
        \subfigure[前背景分离蒙版]{ \includegraphics[width=0.45\textwidth]{../code/mask3.jpg} }
        \subfigure[方向图]{ \includegraphics[width=0.45\textwidth]{../code/lo3.jpg} }
        \subfigure[脊线增强结果]{ \includegraphics[width=0.45\textwidth]{../code/filtered3.jpg} }
    \end{center}
    \caption{latent处理结果}
    \label{fig:img2}
\end{figure}

\end{document}
