\documentclass[a4paper]{article}
\usepackage{xeCJK}
\usepackage{url}
\usepackage{listings}
\usepackage{graphicx}
\usepackage{subfigure}
\setCJKmainfont{STXihei}

\begin{document}
\bibliographystyle{unsrt}
\title{数字图像处理\quad 小作业5}
\author{罗云鹏\ 自64\ 2016011470}
\maketitle

\section{汽车移动}
Matlab工具箱中已有运动模糊函数,其原理是生成一个沿着给定方向的均值滤波器。
为了实现向某一方向运动的效果,将此滤波器从向两边平均改为向一个方向取值做平均。
具体代码如下
\begin{lstlisting}
    h = 2 * fspecial('motion',len * 2,0);
    h(1: len) = 0;
\end{lstlisting}
使中心点以左的滤波器取值为0,调整使其总和约为1。

使用图像处理软件得到汽车区域的mask,将汽车部分图像取出。
对此部分图像做运动模糊,结果中不为0的部分即为汽车会扫过的区域。
对原图做运动模糊,将汽车会扫过部分保留,其余部分使用未处理过的原图背景。
观察真实拍摄的模糊图像,可以看出汽车最后所在位置的图像较为清晰,因此将汽车区域的结果和原图结果求平均,作为最终的结果。

\begin{figure}[!htp]
    \begin{center}
        \includegraphics[width=4in]{../code/carmotion.jpg}
    \end{center}
    \caption{汽车运动模拟}
    \label{fig:carmotion}
\end{figure}

\section{相机随汽车移动}
与汽车移动相似,对全图做左向的运动模糊,并将汽车部分保留为原图,即可模拟出相机随着汽车移动的效果。

\begin{figure}[!htp]
    \begin{center}
        \includegraphics[width=4in]{../code/bgmotion.jpg}
    \end{center}
    \caption{相机随汽车运动模拟}
    \label{fig:bgmotion}
\end{figure}

\section{添加噪声并恢复}
由于两种模拟移动的处理区域不规则,使用逆滤波或者维纳滤波是很难恢复出原图的。
多次试验后发现,仅在移动距离很小的时候,才能勉强看出原图的痕迹。
如果移动距离过大,则恢复结果中几乎看不出原图的痕迹。
因此,为了完成作业要求,对全图做运动模糊代替局部模糊,用于检验维纳滤波和逆滤波的效果。

使用维纳滤波和逆滤波效果见后图。
可以看出,在没有噪声时,两种算法效果相差不大。
当有噪声时,维纳滤波显著好于逆滤波,但如果噪声过强,维纳滤波也不能得到较好的恢复图像。

\begin{figure}[!htp]
    \begin{center}
        \subfigure[逆滤波 高斯噪声方差为0]          { \includegraphics[width=0.4\textwidth]{../code/invH0.jpg} }
        \subfigure[维纳滤波 高斯噪声方差为0]        { \includegraphics[width=0.4\textwidth]{../code/wnr0.jpg} }
        \subfigure[维纳滤波 高斯噪声方差为0.00001]  { \includegraphics[width=0.4\textwidth]{../code/wnr1e-05.jpg} }
        \subfigure[逆滤波 高斯噪声方差为0.00001]    { \includegraphics[width=0.4\textwidth]{../code/invH1e-05.jpg} }
        \subfigure[维纳滤波 高斯噪声方差为0.0001]   { \includegraphics[width=0.4\textwidth]{../code/wnr0.0001.jpg} }
        \subfigure[逆滤波 高斯噪声方差为0.0001]     { \includegraphics[width=0.4\textwidth]{../code/invH0.0001.jpg} }
        \subfigure[维纳滤波 高斯噪声方差为0.001]    { \includegraphics[width=0.4\textwidth]{../code/wnr0.001.jpg} }
        \subfigure[逆滤波 高斯噪声方差为0.001]      { \includegraphics[width=0.4\textwidth]{../code/invH0.001.jpg} }
        \subfigure[维纳滤波 高斯噪声方差为0.01]     { \includegraphics[width=0.4\textwidth]{../code/wnr0.01.jpg} }
        \subfigure[逆滤波 高斯噪声方差为0.01]       { \includegraphics[width=0.4\textwidth]{../code/invH0.01.jpg} }
        \subfigure[维纳滤波 高斯噪声方差为0.1]      { \includegraphics[width=0.4\textwidth]{../code/wnr0.1.jpg} }
        \subfigure[逆滤波 高斯噪声方差为0.1]        { \includegraphics[width=0.4\textwidth]{../code/invH0.1.jpg} }
    \end{center}
    \caption{不同程度噪声滤波结果}
\end{figure}

\end{document}
