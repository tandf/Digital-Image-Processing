\documentclass[a4paper]{article}
\usepackage{xeCJK}
\usepackage{url}
\setCJKmainfont{STXihei}

\begin{document}
\bibliographystyle{unsrt}
\title{数字图像处理\quad 小作业1 \\
\Huge 人脸识别系统
}
\author{罗云鹏 自64 2016011470}
\maketitle
\nocite{cite1}\nocite{cite2}\nocite{cite3}
流程图如图\ref{fig:p1}所示。

\begin{figure}[!htp]
    \begin{center}
        \includegraphics[width=\textwidth]{数图作业-流程图.png}
    \end{center}
    \caption{流程图}
    \label{fig:p1}
\end{figure}

人脸识别系统的系统框图如图\ref{fig:p2}所示。

\begin{figure}[!htp]
    \begin{center}
        \includegraphics[width=\textwidth]{数图作业-系统框图.png}
    \end{center}
    \caption{系统框图}
    \label{fig:p2}
\end{figure}

\section{人脸定位}
人脸定位模块输入一张图片,可定位出其中的人脸位置。

常见的方法如OpenCV中的haar分类器,基于 Viola-Jones 算法,但准确度和召回率曲线较差。
也有其他方法,如微软亚洲研究院14年提出的一种算法,同时完成人脸定位和特征点识别两个功能,提高了效率。

\section{特征点定位}
特征点定位模块输入待识别的人脸,输出其中的特征点。

\section{人脸对齐}
输入待识别的人脸及其中的特征点,根据特征点的位置做变形,得到方向正确的人脸。

\section{人脸校验及人脸识别}
人脸校验和人脸识别模块都输入带有特征点的人脸图像,输出识别结果。
其中,人脸校验功能判断输出的人脸图像是否为某人,而人脸识别功能判断出图像中的是谁。

\bibliography{Report}

\end{document}
