\documentclass[a4paper]{ctexart}
\pagestyle{plain}
\usepackage{url}
\usepackage{listings}
\usepackage{enumerate}
\usepackage{graphicx}
\usepackage{subfigure}
\usepackage{float}
\setCJKmainfont{STXihei}
\usepackage{geometry}
\geometry{left=1.5in, right=1.5in, top=1.5in, bottom=1.5in}
\CTEXsetup[format={\large\bfseries}]{section}
\CTEXsetup[format={\bfseries}]{subsection}

\begin{document}
\bibliographystyle{unsrt}
\title{数字图像处理\quad 综合作业2}
\author{罗云鹏\ 自64\ 2016011470}
\maketitle
\section{实验要求}
此次综合作业要求从冠状动脉的增强图像中提取中心线,主要包含以下步骤:

\begin{enumerate}
\item 图像二值化
\item 空洞填充
\item 中心线提取(图像细化)
\item 分叉点、端点检测
\item 断连分支重连、孤立分支删除
\item 中心线分支模型构建
\end{enumerate}

\section{图像二值化}
此步骤在题目给出的 coronary\_refine.m 中已经完成,据测试效果较好,若改变其阈值$0.5 * intmax('uint16')$不会有明显的改进,故未作改动。

\section{空洞填充}
给出的冠状动脉增强图像数据中,凭肉眼观察很难看出有空洞存在。
使用图像闭操作可去除潜在的空洞。

在matlab工具箱中,有现成的函数$imclose()$可使用。

图像二值化及填补空洞结果如图\ref{fig:1}

\begin{figure}[H]
    \begin{center}
        \subfigure[ours\_054\_c1.mha]{
            \includegraphics[width=\textwidth]{../pic/01.png}}
        \subfigure[ours\_066\_c1.mha]{
            \includegraphics[width=\textwidth]{../pic/07.png}}
    \end{center}
    \caption{bin and fill}
    \label{fig:1}
\end{figure}

\clearpage

\section{中心线提取(图像细化)}
在中心线提取这一步中,可先提取出冠动脉二值图像的骨架作为中心线。

在matlab工具箱中,有现成的函数$bwskel()$可使用。

图像细化结果与填补空洞后图像对比如图\ref{fig:2}

\begin{figure}[H]
    \begin{center}
        \subfigure[ours\_054\_c1.mha]{
            \includegraphics[width=\textwidth]{../pic/02.png}}
        \subfigure[ours\_066\_c1.mha]{
            \includegraphics[width=\textwidth]{../pic/08.png}}
    \end{center}
    \caption{fill and thin}
    \label{fig:2}
\end{figure}

\clearpage

\section{分叉点、端点检测}
在matlab工具箱中,有现成的函数$bwmorph3$可用于提取分支点和端点。

图像端点与填补空洞后图像对比如图\ref{fig:3},
图像分叉点与填补空洞后图像对比如图\ref{fig:4}

\begin{figure}[H]
    \begin{center}
        \subfigure[ours\_054\_c1.mha]{
            \includegraphics[width=\textwidth]{../pic/03.png}}
        \subfigure[ours\_066\_c1.mha]{
            \includegraphics[width=\textwidth]{../pic/09.png}}
    \end{center}
    \caption{fill and end points}
    \label{fig:3}
\end{figure}

\begin{figure}[H]
    \begin{center}
        \subfigure[ours\_054\_c1.mha]{
            \includegraphics[width=\textwidth]{../pic/04.png}}
        \subfigure[ours\_066\_c1.mha]{
            \includegraphics[width=\textwidth]{../pic/10.png}}
    \end{center}
    \caption{fill and bifucation}
    \label{fig:4}
\end{figure}

\clearpage

\section{断连分支重连、孤立分支删除}
\subsection{断连分支重连}
细化后的图像中,存在一些断开的分支。
对于端点接近的分支,可以将端点连接起来,达到断连分支重连的效果。
为防止同一个端点连接多个端点,应寻找所有未连通的端点中距离最近的两个相连,并重新计算端点。
若最近的两个端点距离过大,则说明所有应当连上的分支都连上了,则停止这一循环过程。

\subsection{孤立分支删除}
使用matlab工具箱提供的$bwlabeln()$函数区分连通域,将整个图像划分为若干个相互连通的中心线点集。
判断各个点集的大小,将过小的点集删除。

\subsection{短分支删除}
中心线中,有部分明显过短的分支,可认为是噪声。
为了将这些分支删除,将图中的分叉点删除,并区分各个连通域,每个连通域都是一个无分叉的分支。
与孤立分支删除中操作类似,判断各个分支的长度,将过短的分支删除掉即可。

连接孤立分支并删除部分分支后图像,与填补空洞后图像对比如图\ref{fig:5}

\begin{figure}[H]
    \begin{center}
        \subfigure[ours\_054\_c1.mha]{
            \includegraphics[width=\textwidth]{../pic/05.png}}
        \subfigure[ours\_066\_c1.mha]{
            \includegraphics[width=\textwidth]{../pic/11.png}}
    \end{center}
    \caption{thin and connected branches}
    \label{fig:5}
\end{figure}

\clearpage

\section{中心线分支模型构建}
经过以上步骤处理过的中心线,删去分叉点,可得到各个分支。
将之前得到的分叉点进行处理,相距较近的分叉点坐标求平均,得到一个唯一的分叉点。
对于每个分支,若有分叉点与其端点相距比较近,则将这个分叉点加入其中。

取每个分支的一个端点,作为分支的起点。
其相邻的26邻域内,属于此分支的下一个点作为分支的第二个点,依此类推,可得到按顺序排列的点集。
此点集即为所求结果,可用于绘制曲线图。

各分支散点图与曲线图如图\ref{fig:6}所示

\begin{figure}[H]
    \begin{center}
        \subfigure[ours\_054\_c1.mha]{
            \includegraphics[width=\textwidth]{../pic/06.png}}
        \subfigure[ours\_066\_c1.mha]{
            \includegraphics[width=\textwidth]{../pic/12.png}}
    \end{center}
    \caption{branch points and line}
    \label{fig:6}
\end{figure}

\clearpage

\end{document}
