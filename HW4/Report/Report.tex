\documentclass[a4paper]{article}
\usepackage{xeCJK}
\usepackage{url}
\setCJKmainfont{STXihei}

\begin{document}
\bibliographystyle{unsrt}
\title{数字图像处理\quad 小作业4}
\author{罗云鹏 自64 2016011470}
\maketitle

\section{2D DFT 函数的实现}
二维平面的傅里叶变换,即先对图像每行做一维傅里叶变换,再对每列做一维傅里叶变换。
具体实现见代码。

\section{三种图像的生成与图形化界面显示}
使用图像化界面,先获取输入的图像种类,并按照图像种类显示各参数的滑动条。
比如正弦波的参数有角度、相位等。
根据用户的输入参数,生成图像并显示。
当用户完成对参数的设置后,点击dft按钮,对图像进行傅里叶变换,并将结果显示在窗口中。

\section{附件说明}
\paragraph{code/dft2.m} 2D DFT 函数
\paragraph{code/code.m} 生成不同参数的三种图形的代码
\paragraph{code/window.m 与 window.fig} 图形化界面的代码

\end{document}
