\documentclass[a4paper]{article}
\usepackage{xeCJK}
\usepackage{url}
\usepackage{listings}
\usepackage{graphicx}
\usepackage{subfigure}
\usepackage{float}
\setCJKmainfont{STXihei}

\begin{document}
\bibliographystyle{unsrt}
\title{数字图像处理\quad 小作业6}
\author{罗云鹏\ 自64\ 2016011470}
\maketitle

\section{中值滤波去除椒盐噪声}
对图片添加不同比例的椒盐噪声,并使用中值滤波去除。
噪声比例为 0.01 和 0.05 时,使用$3 \times 3$大小的中值滤波可较好地去除噪声。
当噪声比例达到 0.1 时,使用$5 \times 5$大小的才能去除噪声。
效果如fig\ref{fig:salt1} fig\ref{fig:salt2} fig\ref{fig:salt3}

\begin{figure}[H]
    \begin{center}
        \subfigure[添加椒盐噪声]{\includegraphics[width=0.45\textwidth]{../salt/salt1.jpg}}
        \subfigure[中值滤波去除噪声]{\includegraphics[width=0.45\textwidth]{../salt/restoration1.jpg}}
    \end{center}
    \caption{噪声比例 0.01}
    \label{fig:salt1}
\end{figure}

\begin{figure}[H]
    \begin{center}
        \subfigure[添加椒盐噪声]{\includegraphics[width=0.45\textwidth]{../salt/salt2.jpg}}
        \subfigure[中值滤波去除噪声]{\includegraphics[width=0.45\textwidth]{../salt/restoration2.jpg}}
    \end{center}
    \caption{噪声比例 0.05}
    \label{fig:salt2}
\end{figure}

\begin{figure}[H]
    \begin{center}
        \subfigure[添加椒盐噪声]{\includegraphics[width=0.45\textwidth]{../salt/salt3.jpg}}
        \subfigure[中值滤波去除噪声]{\includegraphics[width=0.45\textwidth]{../salt/restoration3.jpg}}
    \end{center}
    \caption{噪声比例 0.1}
    \label{fig:salt3}
\end{figure}

\section{人物分割}
人物分割算法重点在于设计蒙板。
通过计算某像素中,绿色占三通道颜色值之和的比例,可以判断此像素是否为绿色背景。
尝试后发现,比例大于0.45则可认为是背景。
得到的蒙板中,人物部分有很多缺陷,对蒙板进行闭操作,去除背景中杂点,并做开操作,补全人物部分,即可得到较好的结果。


\begin{figure}[H]
    \begin{center}
        \subfigure[原图]{\includegraphics[width=0.45\textwidth]{../crop/1crop/00001.jpg}}
        \subfigure[人物分割]{\includegraphics[width=0.45\textwidth]{../crop/result/1crop/00001.jpg}}
    \end{center}
    \caption{00001.jpg}
    \label{fig:crop1}
\end{figure}

\begin{figure}[H]
    \begin{center}
        \subfigure[原图]{\includegraphics[width=0.45\textwidth]{../crop/2crop/00096.jpg}}
        \subfigure[人物分割]{\includegraphics[width=0.45\textwidth]{../crop/result/2crop/00096.jpg}}
    \end{center}
    \caption{00096.jpg}
    \label{fig:crop2}
\end{figure}

\begin{figure}[H]
    \begin{center}
        \subfigure[原图]{\includegraphics[width=0.45\textwidth]{../crop/3crop/00465.jpg}}
        \subfigure[人物分割]{\includegraphics[width=0.45\textwidth]{../crop/result/3crop/00465.jpg}}
    \end{center}
    \caption{00465.jpg}
    \label{fig:crop3}
\end{figure}

\end{document}
