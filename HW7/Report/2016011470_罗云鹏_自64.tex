\documentclass[a4paper]{article}
\usepackage{xeCJK}
\usepackage{url}
\usepackage{listings}
\usepackage{graphicx}
\usepackage{subfigure}
\usepackage{float}
\setCJKmainfont{STXihei}

\begin{document}
\bibliographystyle{unsrt}
\title{数字图像处理\quad 小作业7}
\author{罗云鹏\ 自64\ 2016011470}
\maketitle

\textbf{\Large 任务目标}
\begin{itemize}
    \item 对左心室进行对齐
    \item 进行主成分分析
    \item 通过改变系数,观察前三个主成分的意义
    \item 对于某个左心室样本,观察不同数目主成分的重建形状
\end{itemize}

前两个任务目标可以使用给出的代码直接完成。

改变系数,可以发现前三个主成分主要决定了左心室的方向,且每个的系数主成分决定了左心室在某个轴方向上旋转的角度。

使用不同数量的主成分进行重建,其结果差别不大。
理论上,数目越多其结果越接近原有图像。

\end{document}
