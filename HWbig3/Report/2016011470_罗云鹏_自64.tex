\documentclass[a4paper]{ctexart}
\pagestyle{plain}
\usepackage{url}
\usepackage{listings}
\usepackage{enumerate}
\usepackage{graphicx}
\usepackage{subfigure}
\usepackage{float}
\usepackage{geometry}
\geometry{left=1in, right=1in, top=1in, bottom=1in}
\CTEXsetup[format={\large\bfseries}]{section}
\CTEXsetup[format={\bfseries}]{subsection}

\begin{document}
\bibliographystyle{unsrt}
\title{数字图像处理\quad 综合作业3}
\author{罗云鹏\ 自64\ 2016011470}
\maketitle

\section{要求}
对给定的射门视频,检测分割足球并追踪轨迹,估计射门球速。

给定的九个视频中,视频0可以使用示例代码完成检测,视频质量高,检测简单,几乎不需要改动代码。
而剩下的八个视频录制角度、清晰度都不同,需要用其他的方法检测。

检测后发现,视频0帧率为24,检测到的轨迹有明显的四个点一组,中间相隔一定距离的情况。
据猜测,这是因为视频遭到了压缩,从30帧率压缩到24帧率。

由于这些原因,算法主要针对后八个视频来设计。

\section{算法原理}
算法原理如下,具体实现见代码。
其中\textit{ tracking\_basic.m} 为基础任务部分,\textit {tracking.m} 为使用了卡尔曼滤波和自动识别初始点的代码。

\subsection{预处理}
使用MATLAB自带的 {\textit{vision.ForegroundDetector}} 检测运动物体,效果相比实例代码较好。
但两个算法都有一个问题:无法区分运动物体是球还是其他物体。
为了解决这一问题,观察处理结果,不难发现,视频每一帧的部分区域都会检测出运动,这一区域可能是踢球的人,可能是背景中的车辆等等。
除此之外,当然还会检测出运动的足球。
两者相比,足球的运动速度快得多,在一个位置不会停留太久,而其他物体则会停留一段时间。

因此,先使用检测器,对视频每一帧做运动物体检测。
检测得到的二值图像求和,除以帧数以达到归一化的效果。
对结果取一阈值,实验发现0.1左右即可。
对于大于阈值的部分,认为是误检测到的物体,其余部分认为是不运动物体或者是快速运动的足球。
此外,还应当进行一些简单的形态学操作,达到去除噪点等效果。
由此,可得到一个预处理蒙板。

\subsection{足球初始位置确定}
从视频第一帧开始,读取一帧,检测其中的运动物体。
使用预处理蒙板去除非球的部分,在剩下的物体中寻找中心。
如果使用蒙板后没有物体留下,则读取下一帧检测其中物体,直到寻找到球的为止。

实验发现,算法可较好地定位到球的位置。

\subsection{轨迹追踪}
与实例代码中相似,对视频每一帧寻找运动物体。
根据上一次的足球位置信息,去除一定范围外的物体,以及预处理时检测出的非球的运动物体,则得到了这一时刻的球的前景图。
根据前景图,即可计算出球的中心位置。
再使用MATLAB提供的卡尔曼滤波器对结果进行矫正,即可得到较为准确的中心位置。

在前景图中进行圆检测,有可能可以检测出足球的大小。
选取一个视频中可靠性最高的那个检测结果,作为足球的测量半径。

\subsection{速度计算及处理}
由轨迹追踪,可以得到足球中心点序列。
通过计算两个中心点之间的欧式距离,根据足球的测量半径,可换算出其真实距离。
根据视频的帧率,可以计算出瞬时速度。
由于轨迹追踪并不是完全准确的,使用统计学方法去除其中不可靠数据:
将相对于速度均值的偏差大于两倍标准差的数据点删除。
这样,可以得到一个较为准确的速度序列,并可以计算出速度均值和最大值。

\clearpage
\section{处理结果}
足球轨迹追踪如图\ref{fig:track}
\begin{figure}[htp]
    \begin{center}
        \includegraphics[width=0.43\textwidth]{../result/track1.png}
        \includegraphics[width=0.43\textwidth]{../result/track2.png}
        \includegraphics[width=0.43\textwidth]{../result/track3.png}
        \includegraphics[width=0.43\textwidth]{../result/track4.png}
        \includegraphics[width=0.43\textwidth]{../result/track5.png}
        \includegraphics[width=0.43\textwidth]{../result/track6.png}
        \includegraphics[width=0.43\textwidth]{../result/track7.png}
        \includegraphics[width=0.43\textwidth]{../result/track8.png}
    \end{center}
    \caption{}
    \label{fig:track}
\end{figure}

\clearpage

射门球速分布,计算得到的有效速度数量,平均速度,最大速度结果如图\ref{fig:result}

\begin{figure}[htp]
    \begin{center}
        \includegraphics[height=0.18\textheight]{../result/hist1.png}
        \includegraphics[height=0.18\textheight]{../result/hist2.png}
        \includegraphics[height=0.18\textheight]{../result/hist3.png}
        \includegraphics[height=0.18\textheight]{../result/hist4.png}
        \includegraphics[height=0.18\textheight]{../result/hist5.png}
        \includegraphics[height=0.18\textheight]{../result/hist6.png}
        \includegraphics[height=0.18\textheight]{../result/hist7.png}
        \includegraphics[height=0.18\textheight]{../result/hist8.png}
    \end{center}
    \caption{}
    \label{fig:result}
\end{figure}

\end{document}
