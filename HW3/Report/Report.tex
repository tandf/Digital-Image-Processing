\documentclass[a4paper]{article}
\usepackage{xeCJK}
\usepackage{url}
\setCJKmainfont{STXihei}

\begin{document}
\bibliographystyle{unsrt}
\title{数字图像处理\quad 小作业3}
\author{罗云鹏 自64 2016011470}
\maketitle

\section{局部统计量的高效计算}
对于过程的解释见源代码文件(./Q1/ex312\_LocalEnhance.m)。

在原有代码中,局部统计量使用的是nlfilter函数计算的,这个方法的时间花费极长。

考虑局部平均值的计算过程,其计算重点在于求得区域内像素的和。
使用积分图像方法可大大加快这一过程。
在查找资料的过程中,发现matlab中有现有的积分图像函数,且有函数针对积分图像的计算结果进行滤波操作。
利用这两个函数可快速计算某像素的局部平均值,如图\ref{fig:local_mean},计算速度大大提高,且计算结果相同。

\begin{figure}[!htp]
    \begin{center}
        \includegraphics[width=4in]{../Q1/local_mean.png}
    \end{center}
    \caption{计算局部平均值}
    \label{fig:local_mean}
\end{figure}

计算局部方差时,可利用工具箱中现有的函数计算,如图\ref{fig:local_std},所示,其计算速度大大加快,且计算结果相同。

\begin{figure}[!htp]
    \begin{center}
        \includegraphics[width=4in]{../Q1/local_std.png}
    \end{center}
    \caption{计算局部方差}
    \label{fig:local_std}
\end{figure}

\section{模拟大光圈}
简单来说,大光圈的效果是使得某一景深的物体清晰地显示出来,其他景深的物体模糊化。
要实现这一效果,一个朴素的方法就是,选定某一区域图像保留,其余部分进行高斯模糊处理。

\subsection{算法思路}
先将图片二值化,并用鼠标选定对焦目标范围。
对范围内的二值化图形做闭运算,获得一个内部封闭的掩模。
将图片做高斯模糊处理,并将掩模内的部分用原图替代,即可得到处理结果,如图\ref{fig:bokeh}。

\begin{figure}[!htp]
    \begin{center}
        \includegraphics[width=2.2in]{../Q2/1.jpg}
        \includegraphics[width=2.2in]{../Q2/1bokeh.jpg}
        \\\vspace{0.1in}
        \includegraphics[width=2.2in]{../Q2/2.jpg}
        \includegraphics[width=2.2in]{../Q2/2bokeh.jpg}
        \\\vspace{0.1in}
        \includegraphics[width=2.2in]{../Q2/3.jpg}
        \includegraphics[width=2.2in]{../Q2/3bokeh.jpg}
        \\\vspace{0.1in}
        \includegraphics[width=2.2in]{../Q2/4.jpg}
        \includegraphics[width=2.2in]{../Q2/4bokeh.jpg}
        \\\vspace{0.1in}
        \includegraphics[width=2.2in]{../Q2/5.jpg}
        \includegraphics[width=2.2in]{../Q2/5bokeh.jpg}
    \end{center}
    \caption{拍摄得到的大光圈效果(左)与部分模糊处理(右)}
    \label{fig:bokeh}
\end{figure}

\subsection{不足与可能的改进方法}
此方法中采用掩模较为粗糙,对焦图像与背景之间分割效果不佳。
可考虑使用边缘检测算法获取物体范围,或者用其他的方法将背景与对焦物体区分开。

背景中采用的模糊是无区别的均匀模糊,但是实际上,大光圈得到的模糊效果是不均匀的。
不难看出,景深与对焦物体相差越大的,模糊程度越高。
可以改进模糊算法,使得离对焦物体越近的位置模糊程度越低,离对焦物体越远的位置模糊程度越高,这样可以粗略地模拟出真实效果。
也可以手动选定不同区域,分别调节模糊程度。

\end{document}
