\documentclass[a4paper]{article}
\usepackage{xeCJK}
\usepackage{url}
\setCJKmainfont{STXihei}

\begin{document}
\bibliographystyle{unsrt}
\title{数字图像处理\quad 小作业2}
\author{罗云鹏 自64 2016011470}
\maketitle

\section{题目一}
\subsection{计时结果}
\begin{figure}[!htp]
    \begin{center}
        \includegraphics[width=2in]{../Q1/result.png}
    \end{center}
    \caption{题目一运行结果}
    \label{fig:Q1-result}
\end{figure}
运行结果如图\ref{fig:Q1-result}所示。

容易看出,使用 for 循环和使用向量方法计算时,其速度相差不大,使用向量方法计算时耗时略短。
而使用 MEX 编程时,运行时间显著较短,但耗时也处于同一个数量级。

\subsection{遇到的问题及解决方法}
\subsubsection{计时不准确,波动大}
可能是电脑处于低负荷运行,运算时间受各种因素印象较大。
采用运行十次求取平均值的方法得到最后结果。

\subsubsection{MEX 编程耗时过长}
当简单地采用for循环的方法编程时,其运行时间竟然比matlab编程还要长。
根据问题优化算法后,将运行时间降低。

\section{题目二}
\begin{figure}[!htp]
    \begin{center}
        \includegraphics[width=2.2in]{../Q2/a.jpg}
        \includegraphics[width=2.2in]{../Q2/ae.jpg}

        \includegraphics[width=2.2in]{../Q2/b.jpg}
        \includegraphics[width=2.2in]{../Q2/be.jpg}

        \includegraphics[width=2.2in]{../Q2/c.jpg}
        \includegraphics[width=2.2in]{../Q2/ce.jpg}

        \includegraphics[width=2.2in]{../Q2/d.jpg}
        \includegraphics[width=2.2in]{../Q2/de.jpg}

        \includegraphics[width=2.2in]{../Q2/e.jpg}
        \includegraphics[width=2.2in]{../Q2/ee.jpg}
    \end{center}
    \caption{添加噪声前后对比}
    \label{fig:Q2-result}
\end{figure}
添加噪声前后效果如图\ref{fig:Q2-result}所示。

\end{document}
